%%%%%%
%%
%%  Don't reorder the reviewer points; that'll mess up the automatic referencing!
%%
%%%%%

\begin{minipage}[b]{2.5in}
  Resubmission Cover Letter \\
  {\it Genetics}
\end{minipage}
\hfill
\begin{minipage}[b]{2.5in}
    Han Li \\
    \emph{and} Peter Ralph \\
  \today
\end{minipage}
 
\vskip 2em
 
\noindent
{\bf To the Editor(s) -- }
 
\vskip 1em

We are pleased to submit a revision of our manuscript, 

\noindent \hspace{4em}
\begin{minipage}{3in}
\noindent
{\bf Sincerely,}

\vskip 2em

{\bf 
Han Li and
Peter Ralph
}\\
\end{minipage}

\vskip 4em

\pagebreak

%%%%%%%%%%%%%%
\reviewersection{AE}

\begin{quote}
    Please understand that incremental changes will not be sufficient.  Adding
    simulations to strengthen key claims will be necessary, particularly addressing
    the impacts of mutation rate and recombination rate variation with more depth,
    the concern regarding PC switching (Reviewer 1), and the concern regarding the
    impacts of variation in missingness by sub-population (Reviewer 2).
\end{quote}

Thanks for the positive feedback and the useful suggestions.
We agree that more extensive exploration 
using simulations would help bolster understanding of the method,
and have now done so.
This took a substantial amount of work, because genome-scale forwards-time simulations
with a large number of loci under selection
is at or beyond the current limits of computation,
depending on the number of individuals simulated.


%%%%%%%%%%%%%%
\reviewersection{1}

\begin{quote}
    The paper is generally well written and clear; it addresses an important
    problem, and clearly makes some progress on it. However, it suffers from having
    no grounding in either theory or empirical demonstration that it really can find
    the structures that are claimed. I find the arguments that it finds inversions
    compelling, though not watertight, and I am not yet convinced that it is finding
    ubiquitous background selection.  To make this claim, significant extra work is
    required.
\end{quote}

\begin{quote}
    In short, the approach is interesting but not sufficiently explored to produce
    compelling evidence for the implications that are claimed.  Putting a large
    amount of effort into simulations may alleviate these concerns somewhat.

    Specific points: What does this method find? I'm concerned about:
    (a) variation in the recombination rate
    and (b) variation in the mutation rate, creating spurious structure.

    The first possibility is that massively varying information quantity
    within windows could lead to a small number of such windows having their
    orientation reversed: that is, PC1 becomes PC2 and vice versa. (Or PC2 and PC3
    could switch). This would lead to such windows having unusual properties and
    hence appearing as evidence of an inversion.

    I do agree with the authors that significant outliers would be found at
    inversions. However, even if the PC switching does not occur, or the model could
    handle it, the evidence for selection is weaker.  If the two types of variation
    described above exist, with no selection, I would still expect a ``continuous
    triangle'' of results (as seen left of Fig 2, top left of Fig 6) with extrema
    described by windows with the most information, and points placed at different
    extremum having low recombination rate (because by chance, these will get an
    approximately fixed local tree, corresponding on average to the genome-wide
    population structure).

    Addressing this is likely quite hard, though the authors may be able to think of
    something that separates these effects from selection.
\end{quote}


\begin{point}{}
\ldots variation in the recombination rate \ldots creating spurious structure.
\end{point}

\reply{
    We address this in two ways.
    This point is addressed by comparing results with windows of different types --
    windows of equal length in bp (or in SNPs) have different lengths in cM;
    since these different choices show nearly identical patterns,
    recombination rate variation cannot be driving the results.
}

\begin{point}{}
\ldots and variation in the mutation rate, creating spurious structure.
\end{point}

\reply{

}

\begin{point}{PC switching}
\ldots could lead to a small number of such windows having their
orientation reversed: that is, PC1 becomes PC2 and vice versa. (Or PC2 and PC3
could switch).
\end{point}

\reply{
    This is a natural concern.
    However, the only point at which we compare PCs in a way that could be sensitive to ordering
    is in determining the window size -- in computing the distance between windows
    we use a measure which is invariant under ordering.
    We have made this more clear by moving the note about flipping signs of PCs
    to the appendix on window choice \llname{ll:moved_pc_note}
    and added more explicit notes about this to \llname{ll:another_pc_point} and \revref.
}

\begin{point}{p6}
``here, we use k=2...''  - you have to show that $k>2$ is the same.
\end{point}

\begin{point}{p15}
``We also found nearly identical results when choosing shorter windows of 1,000 SNPs'' - again, show this.
\end{point}

\begin{point}{p15}
 ``or choosing windows of equal length in base pairs rather than SNPs'' - once again.
\end{point}

\begin{point}{}
Using 2 PCs is common practice: only if this is the end of an analysis and the
PCA was done for visualisation. Here you are using it for something so should
keep all the relavant PCs.
\end{point}

\reply{
    This is a good point; the question is which the ``relevant'' PCs are.
    \citet{novembre2008interpreting} showed that under isolation by distance,
    the top two PCs should reflect the two-dimensional nature of the range,
    and higher PCs are generally much less interpretable;
    we used $k=2$ with this in mind.
    We have changed this sentence \revref.
}

\begin{point}{}
I'm surprised that PCAdmix isn't referenced. It is using a very similar
method, albeit with different goals. In particular, the approach of placing all
points into a single, genome-wide PC space solves many of the problems that this
approach has (though I agree there may be benefits to the approach described here)
\end{point}


%%%%%%%%%%%%%%
\reviewersection{2}

\begin{quote}
    This is an interesting and well written paper. It was a pleasant read. I have three main general
    comments:
\end{quote}


\begin{point}{Related work:} 
The authors provide an introduction of the main concepts, as well as some
intuition of what the method is doing and how, but I found comparison to previous approaches
to be somewhat missing. To some extent, this is due to the fact that the main goal of their
analysis is somewhat vaguely ``finding heterogeneity'', which leads to the applications of
detecting chromosomal inversions and evidence for background selection. It would help to
have a well defined set of hypotheses, test the method’s accuracy  using simulation (see next
comment), and compare to previous efforts in similar domains.
\end{point}

\reply{
    First: we think that ``finding heterogeneity'' is in fact a well-defined goal,
    although it was not that well-defined in the paper;
    we have hopefully improved on this in the Introduction \revref.
    Expanding a bit more:
    We strongly agree that methods that seek to test well-defined hypotheses
    are extremely useful and powerful.
    We also feel that methods for visualization and exploration are also useful --
    a primary example here being PCA.
    If PCA is useful -- and we think that it is --
    then it should be important to also know how much the thing that PCA is summarizing
    varies along the genome,
    in the same way that knowing the mean of some quantity in a population 
    is only of limited usefulness
    without also knowing the corresponding populaion variance.
}

\begin{point}{Validation:}
In several occasions, the authors seem to introduce a potential problem in their
approach, and provide a solution to it. This is generally rather intuitive, but it would really help
to have simulations of some sort to show that the issue arises and leads to a problem, and that
their approach does address the specific problem.
\end{point}



\begin{point}{}
The use of weighted PCA to cope with unbalanced sample size could be better demonstrated.
Although the current explanation makes intuitive sense, this approach does not seem to be
used in previous work. The authors could design a simulation that supports their approach.
\end{point}

\begin{point}{}
It is conceivable that some subpopulations will have more missingness in some windows. That
may skew the resulting PCs by selecting different sample sizes for the different windows (as
discussed in Appendix B) . This could distort the PCs, so that variation reflects underlying
variation in missingness. Would be good to discuss this potential issue and provide simulations.
\end{point}

\begin{point}{Appendix A:}
when using jackknife to estimate variance, each window is being divided in 10
``independent'' resampling units. Due to LD, these 10 blocks are likely correlated, which would
bias the estimates of variance. This is probably not a problem because both signal and noise
could be equally biased, but the authors may want to consider this potential issue. I wonder if
the correlation with recombination rate may be partially explained by this.
\end{point}

\begin{point}{}
Is it possible to explain the results of Figure 6 just considering neutral variation in local
ancestry due to recent admixture? This may explain why ancestry seems to explain a fair
amount of variance in the lower plots of Fig 6. Local PCA has been previously used by others to
detect local ancestry blocks, e.g. see the PCAdmix approach by Brisbin et al. The authors
discuss the possibility that admixture is driving the differentiation, but do not test whether their
observations agree with neutrality.
\end{point}

\begin{point}{}
``to remove the effect of artifacts such as mutation rate variation, we also rescale each
approximate covariance matrix to be of similar size (precisely, so that the underlying data
matrix has trace norm equal to one''. This potential issue is a bit unclear to me, since I would
expect that scaling the volume of local trees would not result in changed distances in PC
space. Perhaps the authors could show via simulations that this creates a problem, and that the
normalization addresses it.
\end{point}

\begin{point}{Figure 7:}
are MDS coordinates correlated with recombination rates in this case?
\end{point}

\begin{point}{Application:}
 is what the authors seem to be proposing not already accounted for by linear
mixed model association approaches? If not, this should be clarified. Either way, this paragraph
could be dropped.
\end{point}

\begin{point}{Introduction:}
 ``it is not necessarily clear what aspects of demography should be included in the
concept.'' I find it a bit weird to describe selection as an ``aspect of demography''. Although it
could be seen as such within a coalescent framework, that seems to be just a useful
representation. The authors may consider rewording`.
\end{point}

\begin{point}{}
Paragraph starting in ``Since the definition...''. The notation is a bit unclear. Please check that it
is clear which PC the text refers to.
\end{point}

\begin{point}{}
Would the authors be able to provide a sense for the directionality of effects in Figure 4? It
would be interesting if the authors tried to further characterize regions that are similar due to
higher recombination rates. E.g. is there more/less density of polymorphisms in these regions?
\end{point}

\begin{point}{Page 13:}
typo: ``figures 6 and 6''.
\end{point}

\begin{point}{}
Typo in abstract, line 6 ``, We show'' -> ``. We show''.
\end{point}

\begin{point}{}
Typo: end of introduction ``an visualization''. The whole sentence is a bit weird. The authors just
stated focus is on clustering, not on looking for outliers, but what does it mean that ``we allow
ourselves to be surprised by unexpected signals in the data''?
\end{point}

\begin{point}{}
``There has been substantial debate over the relative impacts of different forms of selection.''
Citation needed.
\end{point}

\begin{point}{}
``Results using larger numbers of PCs were nearly identical''. It would be interesting to have a
supplementary table.
\end{point}

\begin{point}{}
Table 1 legend seems a bit redundant. Columns are self-explanatory.
\end{point}

\begin{point}{}
It would help to have numbered lines and references.
\end{point}

