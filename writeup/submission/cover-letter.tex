To the editor(s) --

We are writing to submit our paper,
``Local PCA shows how population structure differs along the genome'',
for consideration in JOURNAL XXX.
The paper brings together several topics that are currently of substantial interest 
to a wide audience:
First, principal components analysis is one of the most widely used descriptive methods 
applied to genomic data;
however, the results depend on the portion of the genome examined, 
and we provide a systematic method to examine this variation.
Second, the three species we study show substantially different patterns
reflective of substantially different biology,
most notably, the role and scale of chromosomal inversions.
Finally,
although we don't develop these ideas,
our method points the way towards move powerful methods for genome-wide association studies.
Furthermore,
the methods are provided as an R package,
to allow others to easily use the method.

In summary, the paper aims to both
provide conceptual unification to a range of observations
(variation along the genome, linked selection, population structure),
and provide a new statistical method,
which leads to biologically interesting observations.


Sincerely,

 Peter Ralph and Han Li

