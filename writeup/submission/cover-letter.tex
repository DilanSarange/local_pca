\documentclass[stdletter,letterpaper,addrfromright,orderfromdateto,dateleft,11pt,noaddrto,sigleft]{newlfm}
\usepackage[hidelinks]{hyperref}

\topmarginskip{-0.35in}
\bottommarginskip{-1.5in}
\leftmarginsize{1in}
\rightmarginsize{1.25in}
\sigskipbefore{0.4in}
\sigskipafter{0in}
\noLines
\nolines
\noHeadline
\noheadline
\signature{Peter Ralph and Han Li}

\namefrom{}
\addrfrom{Peter Ralph \\ Fenton Hall \\ University of Oregon \\ Eugene, OR 97403-1222 USA \\ \textsc{phone:} (541) 346-5530 \\ \textsc{fax:} (541) 346-0987 \\ \href{mailto:plr@uoregon.edu}{plr@uoregon.edu}}
\phonefrom{(541) 346-5530}
\faxfrom{(541) 346-0987}
\emailfrom{plr@uoregon.edu}

% Department of Mathematics
% Fenton Hall
% University of Oregon
% Eugene, OR 97403-1222 USA
% Phone: 1-541-346-4705
% FAX 1-541-346-0987

\greetto{To the editor(s) --}
\closeline{Sincerely,}

\begin{document}

\begin{newlfm}


% A cover letter MUST include:
% (1) name, address, telephone number, fax number, and e-mail address of author responsible for correspondence regarding the manuscript;
% (2) paragraph highlighting the main points of the work;
% (3) statement that the manuscript has been seen and approved by all listed authors;
% (4) a list of potential referees (referees' names, institution, and e-mail), and, if desired, a reasonable list of individuals with potential conflict, as we work very hard to avoid sending manuscripts to competitors;
% (5) status of any statements of personal communication or other permissions needed (any data presented as unpublished results from individuals other than the authors require permission for use);
% (6) statement regarding databank submission of data; and
% (7) assurance that all gene/protein names and symbols used in the paper adhere to approved nomenclature guidelines for specific species. 


We are writing to submit our paper,
``Local PCA Shows How the Effect of Population Structure Differs Along the Genome'',
for consideration in \textit{PLOS Genetics}.
In this paper, we set out to describe how background patterns of relatedness 
(colloquially, ``population structure'')
vary along the genome at a scale larger than correlations induced by linkage.
It was not certain that we would find significant large-scale heterogeneity in patterns of relatedness;
in the field we tend to think of neutral, background population structure as a single thing,
uniform across the genome except at a few isolated loci under selection.
What we found was not only substantial variation at chromosomal scales,
but different underlying causes in the three species we consider.
The focus of our paper is on the method we developed to identify and visualize such variation in patterns of relatedness.
We believe our method represents a fundamental advance because it lets the data speak for itself,
providing an unbiased view of what are the most important variation in patterns of relatedness,
rather than looking for specific signals.
The method is therefore complementary to more common methods that identify small regions of the genome
that are outliers for particular summaries of relatedness (e.g., $F_{ST}$-based scans for selection).
We think the paper will be of interest to readers of \textit{PLOS Genetics}
because the method fills an important gap in the genomicist's toolkit,
that is tailored to describing variation in modern, spatially sampled datasets,
and because it sheds light on the roles of linked selection and chromosomal inversions in 
establishing modern genetic diversity.
% Our results are consistent with what is known about linked selection and chromosomal inversions
% (a possible suprise being no evidence for segregating inversions in \textit{Medicago truncatula});
% the strength of the biological results lie in the unified framework that allows 
% flexible discovery and comparison between the three species.

Potential reviewers for this paper could include:
John Novembre (U Chicago, \href{mailto:jnovembre@uchicago.edu}{jnovembre@uchicago.edu}),
Joe Pickrell (NY Genome Center, \href{mailto:jkpickrell@nygenome.org}{jkpickrell@nygenome.org}),
Alkes Price (Harvard, \href{mailto:aprice@hsph.harvard.edu}{aprice@hsph.harvard.edu}),
or Andy Clark (Cornell, \href{mailto:ac347@cornell.edu}{ac347@cornell.edu}).

If the reviewers would like to read a version of the paper that is single-spaced and has figures included, 
they can download it from \url{http://biorxiv.org/content/early/2016/08/21/070615}.

The paper has been seen and approved by both authors.  
All data we use is publically available, and so no permissions or databank submissions are needed;
however, all source code and substantial derived statistics are available at 
\href{https://github.com/petrelharp/local_pca}{our github page}.
We adhere to all relevant nomenclature guidelines.




\end{newlfm}
\end{document}  


Our paper brings together several topics that are currently of substantial interest 
to a wide audience: 
Methods that search for functional variants in genomic data fundamentally search for regions of the genome
showing unusual patterns of relatedness:
divergence between populations in the case of scans for selection,
or association with a phenotype in the case of GWAS.
In this paper,
we provide a method to visualize larger-scale variations in patterns of relatedness,
show that this variation can be substantial, and is likely the result of both selection and structural variation.
The method represents a fundamental advance because it lets the data speak for itself,
providing an unbiased view of what are the most important variation in patterns of relatedness,
rather than looking for specific signals.
The method is also more suitable to modern, spatially sampled datasets that may not be easily separable into a small number of populations.
On the other hand, the method will be familiar to readers as it combines two often-used approaches: 
visualization with PCA and plots along the genome.
The biological conclusions are also of substantial interest,
as we find that the three species we study differ fundamentally in terms of the influence of linked selection
and the scale and importance of chromosomal inversions,
in particular finding strong evidence to suggest that \textit{Medicago truncatula} harbors no inversions segregating at any appreciable frequency.
